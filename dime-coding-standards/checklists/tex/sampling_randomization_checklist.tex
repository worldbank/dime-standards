
\documentclass{tufte-handout}
\documentclass{tufte-handout}

\usepackage{librecaslon}


\usepackage{fancyhdr}
\usepackage{hyperref}
\usepackage{tcolorbox} % needed for the text boxes
\usepackage{xcolor}
\usepackage{setspace}
\usepackage{graphics}
\hypersetup{
	colorlinks=true,
	linkcolor=blue,
	urlcolor=cyan,
}
% Set header and footer
\pagestyle{fancy}
\fancyhf{}
\lfoot{\includegraphics[height=1cm,keepaspectratio]{../../img/i2i}}
\cfoot{\includegraphics[height=1cm,keepaspectratio]{../../img/wb}}
\rfoot{\includegraphics[height=1cm,keepaspectratio]{../../img/analytics}}

% Put checkbox to the left of the text
\def\LayoutCheckField#1#2{% label, field
	#2 #1%
}

% Line spacing
\onehalfspacing

% Define DIME Analytics visual identity colors
\definecolor{fontcolor}{HTML}{7A0569}

\titleformat{\section}%
{\Large\rmfamily\bf\color{fontcolor}}% format applied to label+text
{\llap{\colorbox{fontcolor}{\parbox{1.5cm}{\hfill\huge\color{fontcolor}\thesection}}}}% label
{2pt}% horizontal separation between label and title body
{}% before the title body
[]% after the title body

\titleformat{\subsection}%
{\large\rmfamily\color{fontcolor}}% format applied to label+text
{}% label
{1.5pt}% horizontal separation between label and title body
{}% before the title body
[]% after the title body

\newcommand{\dimeCheckBox}[1]{\CheckBox[height=0.01cm, width=0.4cm, bordercolor=gray]{#1}}
\newcommand{\dimeTextField}[3]{\TextField[name=#1, height=0.3cm, width=#2, bordercolor=gray]{#3}}


\newcommand{\titleBox}[1]{
	\begin{tcolorbox}
		[colframe = fontcolor,
		colback = fontcolor,
		sharp corners,
		halign = flush center,
		valign = center,
		height = 0.3\textwidth,
		after skip = 1cm]
		#1
	\end{tcolorbox}
}



\begin{document}
\thispagestyle{firstpage}

    \begin{fullwidth}
		
		\titleBox{
			\textcolor{white}{\LARGE{\textbf{DIME Analytics \\ Peer Code Review - Sampling and Random Treatment Assignment}} \\
				}
		}

        \section*{Reviewer Details}
        \dimeTextField{reviewer}{6cm}{Reviewer Name:}  
        \dimeTextField{coder}{6cm}{ Coder Name:}  
		
	\vspace{0.5cm}

        \textbf{Note:} Please complete this checklist \textbf{only if} the submission includes \textbf{sampling and/or randomization} tasks.
		
\section*{Sampling and Randomization Tasks}
This checklist highlights key aspects to review in your partner's \textbf{sampling and randomization} code.  
\begin{itemize}
    \item \textbf{Sampling} refers to defining the sample frame, i.e., selecting which units (e.g., individuals, households) will be included in the study.
    \item \textbf{Randomization} is the process of assigning treatment and control groups within the sampled units.
\end{itemize}
Once completed, please submit it as an attachment along with \href{https://survey.wb.surveycto.com/collect/code_review_summary?caseid=}{this form}.


\section*{Sampling Checklist}
This section focuses on reviewing the code that performs \textbf{sampling}.

\subsection*{Setting Parameters}
\begin{itemize}
    \item[] \dimeCheckBox{The script explicitly sets the software version (e.g., \texttt{version} in Stata).}
    \item[] \dimeCheckBox{A random seed is set for reproducibility (e.g., \texttt{set seed} in Stata or \texttt{set.seed()} in R). }
    \begin{itemize}
                \item[]\dimeCheckBox{If yes, the seed values are set using an externally generated, unique seed (e.g., from \href{http://bit.ly/stata-random}{random.org})}
            \end{itemize}     \item[] \dimeCheckBox{The dataset is sorted by a unique identifier.}
\end{itemize}

\subsection*{Sampling Methdology}
    \begin{itemize}
        \item[] \dimeCheckBox{The sampling strategy is clearly documented, including inclusion/exclusion criteria.}
        \item[] \dimeCheckBox{The sampling method (e.g., stratified, clustered, or simple random sampling) is explicitly stated.}
        \item[] \dimeCheckBox{Handling of unequal cluster sizes is clearly specified and justified.}
        \item[] \dimeCheckBox{Probability weights, if used, are calculated and stored.}
    \end{itemize}


\subsection*{Defining the Sampling Frame}
    \begin{itemize}
        \item[] \dimeCheckBox{The dataset contains necessary variables for sampling, such as clusters and strata.}
        \item[] \dimeCheckBox{\textbf{Advanced check:} Clusters, strata, and IDs do not contain any personally identifiable information (PII).}
        \item[] \dimeCheckBox{\textbf{Advanced check:} The resulting dataset is checked for stability using commands like \texttt{datasignature}, file hashing, \texttt{assert}, and/or \texttt{isid, sort}.}
    \end{itemize}
    

\subsection*{Sampling Execution}
\begin{itemize}
    \item[] \dimeCheckBox{The script outputs a dataset containing a categorical variable that clearly marks sampled and non-sampled groups.}
    \item[] \dimeCheckBox{Final sampling results are stable and reproducible across multiple runs.}
    \item[] \dimeCheckBox{If the sample is finalized for field use, a logic switch prevents accidental overwriting.}
    \item[] \dimeCheckBox{All outputs include codebooks, value labels, and documentation.}
\end{itemize}


\section*{Randomization Checklist}
This section focuses on reviewing the code that performs \textbf{random treatment assignment}.

\subsection*{Setting Parameters}
\noindent
Verify that the coding environment is properly configured to ensure reproducibility.

\begin{itemize}
    \item[] \dimeCheckBox{The script specifies the software version (e.g., \texttt{version} in Stata or package version control in R).}
    \item[] \dimeCheckBox{A random seed is set for reproducibility (e.g., \texttt{set seed} in Stata or \texttt{set.seed()} in R). }
    \begin{itemize}
                \item[]\dimeCheckBox{If yes, the seed values are set using an externally generated, unique seed (e.g., from \href{http://bit.ly/stata-random}{random.org})}
            \end{itemize}
    \item[] \dimeCheckBox{The dataset is sorted by a unique identifier before randomization to maintain consistency across runs.}
\end{itemize}


\subsection*{Random Treatment Assignment}
\begin{itemize}
    \item[] \dimeCheckBox{Treatment assignment is conducted using a script (not manual methods).}
    \item[] \dimeCheckBox{The randomization method is well-documented (e.g., simple, stratified, or block randomization).}
    \item[] \dimeCheckBox{A categorical variable is created to assign treatment and control groups, with appropriate labels.}
    \item[] \dimeCheckBox{No new observations are created during the treatment assignment process.}
    \item[] \dimeCheckBox{Randomization is reproducible and produces consistent results across multiple runs.}
    \item[] \dimeCheckBox{The output includes a dataset with treatment assignments and relevant documentation.}
    \item[] \dimeCheckBox{If the treatment assignment is finalized for use, a logic switch prevents accidental overwriting.}
\end{itemize}
		
	\end{fullwidth}
\end{document}



