
\documentclass{tufte-handout}
\documentclass{tufte-handout}

\usepackage{librecaslon}


\usepackage{fancyhdr}
\usepackage{hyperref}
\usepackage{tcolorbox} % needed for the text boxes
\usepackage{xcolor}
\usepackage{setspace}
\usepackage{graphics}
\hypersetup{
	colorlinks=true,
	linkcolor=blue,
	urlcolor=cyan,
}
% Set header and footer
\pagestyle{fancy}
\fancyhf{}
\lfoot{\includegraphics[height=1cm,keepaspectratio]{../../img/i2i}}
\cfoot{\includegraphics[height=1cm,keepaspectratio]{../../img/wb}}
\rfoot{\includegraphics[height=1cm,keepaspectratio]{../../img/analytics}}

% Put checkbox to the left of the text
\def\LayoutCheckField#1#2{% label, field
	#2 #1%
}

% Line spacing
\onehalfspacing

% Define DIME Analytics visual identity colors
\definecolor{fontcolor}{HTML}{7A0569}

\titleformat{\section}%
{\Large\rmfamily\bf\color{fontcolor}}% format applied to label+text
{\llap{\colorbox{fontcolor}{\parbox{1.5cm}{\hfill\huge\color{fontcolor}\thesection}}}}% label
{2pt}% horizontal separation between label and title body
{}% before the title body
[]% after the title body

\titleformat{\subsection}%
{\large\rmfamily\color{fontcolor}}% format applied to label+text
{}% label
{1.5pt}% horizontal separation between label and title body
{}% before the title body
[]% after the title body

\newcommand{\dimeCheckBox}[1]{\CheckBox[height=0.01cm, width=0.4cm, bordercolor=gray]{#1}}
\newcommand{\dimeTextField}[3]{\TextField[name=#1, height=0.3cm, width=#2, bordercolor=gray]{#3}}


\newcommand{\titleBox}[1]{
	\begin{tcolorbox}
		[colframe = fontcolor,
		colback = fontcolor,
		sharp corners,
		halign = flush center,
		valign = center,
		height = 0.3\textwidth,
		after skip = 1cm]
		#1
	\end{tcolorbox}
}



\begin{document}
\thispagestyle{firstpage}
    \begin{fullwidth}
    


        \titleBox{
            \textcolor{white}{\LARGE{\textbf{DIME Analytics \\ Peer Code Review Checklist: Stata}} \\[5pt]}
        }
% Restore default margins after the first page


        \section*{Reviewer Details}
        \dimeTextField{reviewer}{6cm}{Reviewer Name:}  
        \dimeTextField{coder}{6cm}{ Coder Name:}  

        \vspace{0.5cm}

The following checklist outlines best practices for writing and reviewing Stata code. Items/Sections marked with an asterisk \texttt{*} are strongly recommended to ensure reproducibility.


\subsection*{Main Do-file Setup*}
\begin{itemize}
    \setlength\itemsep{-0.1em}
	\item[] \dimeCheckBox{Sets core configurations (\texttt{version}, \texttt{matsize}, \texttt{varabbrev}) directly or via a wrapper (e.g., \texttt{ieboilstart}).}
	\item[] \dimeCheckBox{Script runs from start to end after changing directory paths in one place only.}
	\item[] \dimeCheckBox{Uses only \textbf{relative paths} (no \texttt{C:/...}).}
	\item[] \dimeCheckBox{Uses forward slashes in file paths for OS compatibility.}
	\item[] \dimeCheckBox{Installs required packages or includes an \texttt{ado} folder with dependencies.}
	\item[] \dimeCheckBox{Sets a \textbf{random seed} for reproducibility.}
	\item[] \dimeCheckBox{The main do-file runs all code files (using \texttt{run} or \texttt{do} to files) without any need to manually run files in a certain order.}
\end{itemize}


\subsection*{Data Management}
\begin{itemize}
    \setlength\itemsep{-0.1em}
    	\item[] \dimeCheckBox{Dataset includes a \textbf{unique ID} and is sorted.}
	\item[] \dimeCheckBox{The same unique ID is used consistently across datasets that share the same unit of observation.}
	\item[] \dimeCheckBox{\textbf{*}Duplicate resolution is stable (does not use \texttt{duplicates drop, force}).}
	\item[] \dimeCheckBox{Does not include PII or sensitive information.}
	\item[] \dimeCheckBox{All variables are clearly labeled.}
	\item[] \dimeCheckBox{Value labels are consistent (e.g., avoiding cases where \texttt{varA: 1 = yes, 0 = no} but \texttt{varB: 1 = yes, 2 = no}).}
	\item[] \dimeCheckBox{Extended missing values are used where applicable (e.g., \texttt{.d} for Do not know, \texttt{.r} for Refuse to answer, etc.).}
	\item[] \dimeCheckBox{\textbf{*}Sorting is consistently and uniquely enforced using \texttt{sort} or \texttt{gsort} before commands that depend on it.}
	\item[] \dimeCheckBox{Avoids saving intermediate datasets unless needed for later use (uses \texttt{tempfile} when appropriate).}
	\item[] \dimeCheckBox{\textbf{*}Saves final dataset only once, avoids repeated overwriting.}
	\item[] \dimeCheckBox{Follows tidy data principles: one row per observation, one column per variable, and one unit of observation per \texttt{.dta} file (e.g., avoid wide-format household member data in a household-level file).}
	\item[] \dimeCheckBox{Avoids interactive commands (\texttt{edit}, \texttt{browse}).}
\end{itemize}

\subsection*{Data Types \& Variables}
\begin{itemize}
    \setlength\itemsep{-0.1em}
	\item[] \dimeCheckBox{String variables are only used when necessary (e.g., proper nouns or alphanumeric IDs).}
	\item[] \dimeCheckBox{Converts categorical strings into labeled numeric variables (e.g., using \texttt{encode}).}
	\item[] \dimeCheckBox{Date variables are stored in proper date formats (e.g., \texttt{\%td}, \texttt{\%tm}).}
\end{itemize}
    
\subsection*{Merge Checks}
\begin{itemize}
    \setlength\itemsep{-0.1em}
    	\item[] \dimeCheckBox{\textbf{*}No \texttt{m:m} merges used.}
	\item[] \dimeCheckBox{Mismatches or dropped observations are explained using \texttt{tab \_merge} and \texttt{assert} checks.}
	\item[] \dimeCheckBox{If any observations are dropped, a clear justification is provided in the code.}
\end{itemize}    

\subsection*{Append Checks}
\begin{itemize}
    \setlength\itemsep{-0.1em}
    	\item[] \dimeCheckBox{Variables being appended are of the same type and structure.}
	\item[] \dimeCheckBox{Avoids \texttt{append, force}.}
	\item[] \dimeCheckBox{Any new variables introduced in appended datasets are properly handled.}
	\item[] \dimeCheckBox{Ensures that the resulting dataset remains uniquely identifiable, either with the original ID or a new combination of variables after the append.}
\end{itemize}    

\subsection*{Code Readability \& Style}
\begin{itemize}
    \setlength\itemsep{-0.1em}
    	\item[] \dimeCheckBox{Uses proper indentation inside loops or programs.}
	\item[] \dimeCheckBox{Uses white space and line breaks (\texttt{///}) for long lines.}
	\item[] \dimeCheckBox{Uses descriptive index names in loops/globals.}
	\item[] \dimeCheckBox{Uses \texttt{\$\{\}} syntax for global macros.}
	\item[] \dimeCheckBox{Comments clearly explain steps and analysis decisions.}
	\item[] \dimeCheckBox{Each section is clearly marked (e.g., \texttt{*** SECTION: Construct Outcomes}).}
	\item[] \dimeCheckBox{Avoids hardcoding values (uses macros).}
	\item[] \dimeCheckBox{Avoids copy-pasting blocks; uses loops or programs where repetitive code appears.}
\end{itemize}    
            
\subsection*{Variable Construction}
\begin{itemize}
    \setlength\itemsep{-0.1em}
    	\item[] \dimeCheckBox{Each variable’s logic aligns with the codebook or documentation.}
	\item[] \dimeCheckBox{Transformations (log, winsorize, unit-standardization, etc.) are justified and explained.}
	\item[] \dimeCheckBox{Categorical variables are properly labeled and encoded.}
	\item[] \dimeCheckBox{Data transformations are verified with \texttt{assert} or summary checks.}	
\end{itemize}    

\subsection*{Collapse / Group-wise Calculations}
\begin{itemize}
    \setlength\itemsep{-0.1em}
    	\item[] \dimeCheckBox{\textbf{*}Data is sorted uniquely before using \texttt{by: egen} or \texttt{by: gen}.}
	\item[] \dimeCheckBox{Aggregations (e.g., using \texttt{collapse}, \texttt{egen}, or group-level calculations) are correct and clearly documented.}
	\item[] \dimeCheckBox{Missing values are handled appropriately during \texttt{collapse} and \texttt{egen}.}
\end{itemize}    

\subsection*{Output \& Logging}
\begin{itemize}
    \setlength\itemsep{-0.1em}
    	\item[] \dimeCheckBox{\textbf{*}Outputs are not copied manually to external files. Instead, they are exported using commands like \texttt{esttab}, \texttt{outreg}, \texttt{asdoc}, \texttt{graph export}, among others.}
	\item[] \dimeCheckBox{Output files are clearly named and saved in dedicated folders.}
	\item[] \dimeCheckBox{Log files are started with \texttt{log using} and closed with \texttt{log close}.}
	\item[] \dimeCheckBox{\textbf{*}Tables are saved in plain text formats (e.g., \texttt{.csv}, \texttt{.txt}, \texttt{.tex}) to ensure compatibility with Git and facilitate version control.}
	\item[] \dimeCheckBox{\textbf{*}Export commands include the \texttt{replace} option to prevent errors if output files already exist.}
\end{itemize}    

\subsection*{Reproducibility \& Documentation*}
\begin{itemize}
    \setlength\itemsep{-0.1em}
	\item[] \dimeCheckBox{Code runs reproducibly from a fresh Stata session.}
	\item[] \dimeCheckBox{README documents required Stata version and packages.}
	\item[] \dimeCheckBox{\texttt{ieboilstart} or equivalent ensures version stability.}
	\item[] \dimeCheckBox{Folder and file structure is documented.}
	\item[] \dimeCheckBox{README specifies the main do-file and highlights which line(s) to update to run the code.}
\end{itemize}                
		\vspace{.6em}


	\end{fullwidth}
\end{document}
